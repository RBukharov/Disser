\chapter*{Введение}							% Заголовок
\addcontentsline{toc}{chapter}{Введение}	% Добавляем его в оглавление

\textbf{Актуальность работы.}
Ленточные конвейеры являются неотъемлемой составляющей транспортных систем горных предприятий в России и за рубежом. 

Технические сложности и особенности работы ленточных конвейеров, такие как проскальзывание ленты, ударные нагрузки в приводе, повышенный износ оборудования, значительный расход электроэнергии при неполной загрузке и работе вхолостую, приводят к поиску решения задач оптимизации и автоматизации процессов работы конвейерных установок, то есть к повышению общей эффективности их эксплуатации. В настоящее время системы управления работой ленточных конвейеров на горных предприятиях решают ограниченный круг задач повышения эффективности работы и увеличения срока службы оборудования, и в большинстве своем реализуют такие функции, как оперативно-диспетчерское управление (пуск, останов, переключение скорости движения ленты), поддержание определенного уровня производительности, сигнализацию и контроль состояния оборудования.

Причинами этого является сложность и особенности технологического процесса (колебания ленты, возникающие при переключении скорости, пуске и останове, возможность проскальзывания ленты при определенных условиях и т. д.), грузопоток, величина которого обычно носит случайный характер, зависимость эффективности работы конвейерной установки от сторонних факторов, например, от погодных условий, а также тяжелые условия работы оборудования автоматизации. Таким образом, разработка и техническая реализация новых решений в области автоматизации и управления конвейерными установками горных предприятий с целью повышения эффективности их эксплуатации является в настоящее время актуальной задачей.
\bigskip

\textbf{Целью работы} является разработка новых решений в области автоматизации и управления для повышения эффективности эксплуатации конвейерного транспорта горных предприятий. В~работе получен новый алгоритм останова конвейера, а также разработана структура и вариант реализации комплексной автоматизированной системы управления магистральным ленточным конвейером, реализующей алгоритмы, которые позволяют оптимизировать исследуемые процессы пуска, останова и изменения скорости движения ленты конвейеров с целью повышения эффективности эксплуатации конвейерного транспорта горных предприятий.
\bigskip

\textbf{Задачи исследования.} Указанная цель определила следующие задачи исследования: 
\begin{itemize}
\item доработка реализованной ранее математической модели ленточного конвейера для возможности исследования процессов торможения;
\item исследование режимов торможения и пуска ленточного конвейера определенного типа;
\item разработка алгоритмов управления исследуемыми режимами работы конвейера;
\item разработка структуры комплексной системы автоматизированного управления, реализующей разработанные алгоритмы;
\item предложение варианта технической реализации комплексной системы автоматизированного управления ленточным конвейером.
\end{itemize}

\textbf{Идея работы} состоит в исследовании процессов останова ленточного конвейера и разработке комплексной системы автоматизированного управления ленточным конвейером для повышения эффективности его эксплуатации.
\bigskip

\textbf{Предмет защиты.} На защиту выносятся следующие новые и содержащие элементы новизны основные положения и результаты:
\begin{enumerate}
  \item доработанная математическая модель ленточного конвейера, позволяющая исследовать процессы торможения;
  \item разработан алгоритм управления остановом конвейера, основанный на предварительном управляемом торможении хвостового барабана, обеспечивающий устранение проскальзывания ленты и повышающий эффективность эксплуатации конвейера;
  \item предложены методические подходы к выбору и расчету параметров алгоритма управления остановом конвейера, позволяющие использовать разработанный алгоритм для конвейеров различных модификаций;
  \item разработана структура комплексной автоматизированной системы управления конвейерной установкой, реализующей несколько алгоритмов управления;
  \item предложен вариант реализации комплексной автоматизированной системы управления конвейерной установкой и разработанных алгоритмов управления с применением современных аппаратных и программных средств.
\end{enumerate}
\bigskip

\textbf{Научная новизна работы состоит в следующем.}
\begin{enumerate}
  \item Доработана математическая модель конвейера, которая позволяет моделировать и исследовать процессы торможения;
  \item Разработаны и реализованы алгоритмы управления, которые позволяют оптимизировать переходные процессы пуска и торможения ленточного конвейера, включающие в себя контроль натяжений ленты конвейера посредством предварительного управляемого торможения барабанов конвейера;
  \item Разработана методика расчета параметров алгоритмов управления, позволяющая применять их для конвейеров различных типов;
  \item Разработана структура комплексной системы управления конвейерной установкой, осуществляющая управление различными процессами, в том числе процессами пуска и торможения конвейера;
  \item Разработано программное обеспечение и подобрано аппаратное обеспечение комплексной системы управления конвейерной установкой.
\end{enumerate}

\bigskip
\textbf{Методы исследования.} При решении поставленных задач использовались методы теории волновых процессов в упругих средах, теоретической механики, классической теории автоматического управления, методы статистического анализа и математического моделирования. Применено современное программное обеспечение Matlab R2012b (8.0.0.783), SIMATIC Step 7, PLCSIM v.5.4 для построения моделей, обработки данных и подтверждения результатов.\\

\textbf{Обоснованность и достоверность научных положений} подтверждается результатами модельных исследований, корректным применением известных методов математического анализа, теории волновых процессов в упругих средах, теоретической механики, теории автоматического управления, теории регулируемого электропривода и достаточной близостью результатов модельных исследований результатам работы реального технологического процесса, взятым с~осциллограмм натурных испытаний и работы в производственных условиях.
\bigskip

\textbf{Практическая значимость.} Разработанные в диссертации методы и алгоритмы управления реализованы на современном промышленном аппаратном обеспечении и могут быть использованы при проектировании новых и совершенствовании действующих систем управления процессами транспортировки грузов на ленточных конвейерах, а также для обучения студентов, повышения квалификации персонала по автоматизации технологических процессов и в научных исследованиях в области автоматизации и управления. Использование реализованных алгоритмов управления в промышленности позволит снизить износ ленты конвейеров, уменьшить расходы электроэнергии, а также минимизировать динамические усилия в ленте, что снизит коэффициент запаса по прочности ленты, позволяя использовать менее дорогие типы ленты и~продлить срок ее службы.
\bigskip

\textbf{Личный вклад автора} заключается в постановке целей и формулировке задач исследований, доработке существующих математических моделей исследуемого объекта, проведении экспериментов с математическими моделями исследуемого объекта, исследовании и моделировании методов управления процессами пуска и торможения конвейера в различных условиях, разработке и отладке алгоритмов управления исследуемыми процессами, разработке методики расчета и выбора параметров алгоритмов управления, проведении модельных исследований разработанных алгоритмов управления и анализе их работы, реализации разработанных алгоритмов управления на современном промышленном оборудовании для автоматизации.
\bigskip

\textbf{Апробация работы.} Основные положения и результаты диссертационной работы докладывались и обсуждались на научной секции XIV Международной студенческой олимпиады по автоматическому управлению BOAC’2011 (НИУ ИТМО, 2011 г.), на ежегодном научном симпозиуме «Неделя горняка» (МГГУ, 2012, 2014 г.), на Международном форуме-конкурсе молодых ученых «Проблемы недропользования» (НМСУ «Горный», 2014 г.), на научных семинарах кафедры «Автоматика и управление в технических системах» (МГГУ, 2012 -- 2014 гг.).

\bigskip
\textbf{Публикации.} По теме диссертации опубликовано 8 научных работ в периодических изданиях и в сборниках научных трудов ~\cite{bib1, bib2, bib3, bib4, bib6, bib7, bib8, bib9}, в том числе 5 работы в изданиях, рекомендованных ВАК Минобрнауки.

\bigskip
\textbf{Объем и структура работы.} Диссертация состоит из~введения, четырех глав, заключения и~четырех приложений. Полный объем диссертации составляет 126~страниц с~58~иллюстрациями и~9~таблицами. Список литературы содержит 81 наименование.

\bigskip
Работа выполнена на кафедре <<Автоматики и управления в технических системах>> Горного института (МГИ) Национального Исследовательского Технологического университета МИСиС.

\clearpage