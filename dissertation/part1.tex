% Обзор научных работ в области автоматизации управления режимами работы конвейерных установок

\chapter{Обзор научно-исследовательских работ, посвященных исследованиям повышения эффективности эксплуатации ленточных конвейеров} \label{chapt1}

\section{Обзор научно-исследовательских работ} \label{sect1_1}

Применение транспортных машин непрерывного действия, в частности ленточных конвейеров, характеризуется низкой эффективностью. Регулирование скорости движения ленты, оптимизация пусковых и тормозных процессов, а также контроль проскальзывания ленты позволяет существенно повысить эффективность эксплуатации транспортных машин. Лента является основным и наименее долговечным элементом ленточного конвейера. Ее стоимость может составлять около половины общей стоимости конвейера, а высокие амортизационные затраты на обслуживание ленты -- немаловажный фактор при определении области применения и экономической эффективности конвейерного транспорта. Поэтому оптимизация режимов работы конвейерной установки, направленная на увеличение срока службы ленты, оказывает существенное влияние на общую эффективность эксплуатации конвейерных установок.  

Проблемам разработки алгоритмов оптимизации работы конвейеров, синтеза регуляторов и~систем управления, проведению модельных и натурных исследований посвящено множество научных работ, разработаны различные структурные схемы систем и алгоритмов регулирования, в том числе оптимальных. Ниже рассмотрены работы следующих авторов: Дмитриевой~В.~В., Мамалыги В. М., Петкова О. Н., Сухарева И. А., Сокотнюка Ю. А., Гершуна С. В., Серикова С. А., Черемушкиной М. С., Мартынова В. В., Городецкого А. В., Кангина В. В. Симонса А..\\

Основные факторы, которые отрицательно сказываются на сроке службы конвейерной ленты~-- это динамические нагрузки при переключении скорости ее движения, а также проскальзывание ленты в различных режимах работы конвейера. Проскальзывание ленты на приводном барабане может возникать при разгоне или торможении конвейера, и это отрицательно влияет не только на срок службы ленты, но и на срок службы привода конвейера. Этот фактор можно исключить или свести к минимуму, применяя различные способы управления, которые основаны на своевременном изменении натяжений в ветвях конвейера таким образом, чтобы соблюдалось условие Эйлера \cite{vdmitriev}, что в свою очередь гарантирует отсутствие проскальзывания ленты. В~режимах пуска конвейера и его номинальной работы целесообразно изменять натяжения в~ветвях конвейера посредством изменения силы тяжести натяжного устройства.\\

В работе \textbf{Дмитриевой В. В.} \textit{«Разработка и исследование системы автоматической стабилизации погонной нагрузки магистрального конвейера»} \cite{vdmitrieva} представлена структура системы автоматического управления и алгоритмов регулирования скоростью движения конвейерной ленты, которые позволяют обеспечить рациональную нагрузку конвейера насыпным грузом при условии обеспечения беспробуксовочной работы главного привода при любом изменении скорости движения ленты. Разработана математическая модель сложной многомерной системы управления движением ленты, динамики натяжного устройства, частотно-управляемого асинхронного привода магистрального конвейера, позволяющая исследовать конвейерную установку как единый объект управления, моделировать технологический процесс транспортирования случайного грузопотока и искать решение задачи стабилизации погонной нагрузки конвейера в классе линейных оптимальных регуляторов. Разработана структура многоконтурной автоматической системы регулирования погонной нагрузки конвейера, стабилизации тяговой способности привода, которая позволяет реализовать алгоритм оптимальной нагрузки конвейера. Разработан алгоритм управления движением ленты, который обеспечивает плавное изменение скорости, соответствующее случайному входному грузопотоку, снижение динамических нагрузок в ленте, уменьшение количества пуско-тормозных режимов, уменьшение холостого пробега ленты.

В работе рассматриваются расчетные схемы для конвейеров с однодвигательным приводом. Разработана комплексная модель стабилизации погонной нагрузки конвейера, включающая подсистемы:
\begin{itemize}
\item модель движения ленточного конвейера и натяжного устройства;
\item модель асинхронного частотно-управляемого короткозамкнутого привода;
\item модель регулятора натяжения;
\item модель регулятора скорости движения ленты, включающая в себя модель шахтного случайного грузопотока, модель задатчика скорости, собственно регулятор и модель устройства, позволяющего определить величину ошибки регулирования.
\end{itemize}

В работе \textbf{Мамалыги В. М.} \textit{«Взаимосвязанная система управления многодвигательными ленточными конвейерами»} \cite{vmamalyga}  предложен способ построения системы автоматического распределения нагрузок в n-двигательном ленточном конвейере с целью их равномерного распределения между приводами, независимо от их расположения и количества, при любой скорости движения конвейера. При выполнении этого условия возможно снизить статический запас прочности ленты и уменьшить стоимость ленты. 

В диссертации разработана структурная схема взаимосвязной системы автоматического управления многобарабанным ленточным конвейером. Особое внимание уделено оптимизации динамических режимов работы конвейера. Рассмотрено ограничение динамических нагрузок при пуске -- рывок и ускорение, это ограничение позволяет снизить износ ленты, уменьшить ее прочность и стоимость, также рассмотрено обеспечение высокой точности регулирования скорости при работе конвейера с постоянной скоростью, что позволяет уменьшить износ и~пробег ленты. Автором введен критерий оптимальности динамических режимов конвейера в виде функционала, в котором учитываются как точность отработки сигнала задания по~скорости, так и ограничения на величины динамического усилия и ускорения. Функционал минимизируется с~использованием интегрального уравнения Винера-Хопфа.\\

В работе \textbf{О. Н. Петкова} \textit{«Разработка и исследование системы автоматического управления скоростью ленточного конвейера по входному грузопотоку»} \cite{opetkov} исследованы карьерные угольные грузопотоки (карьер “Трояново-3”, Болгария) и определены их вероятностные характеристики: корреляционная функция определена в виде:
$$
R(\tau) = \sigma^2 e^{-\alpha|\tau|}\cos\omega_0 \tau,
$$
что соответствует спектральной плотности вида 
$$
S(\omega) = \frac{1}{2\pi} \sigma^2 \alpha \left[ \frac{1}{\alpha^2 + (\omega - \omega_0)^2} +\frac{1}{\alpha^2 + (\omega + \omega_0)^2} \right],
$$
где $ \sigma^2 $ -- дисперсия случайного процесса, $ \alpha $ -- параметр затухания, $ \omega_0 $ -- резонансная частота.

В работе установлено, что законы распределения длительностей поступления и отсутствия грузопотока имеют экспоненциальный характер. Для системы регулирования скорости ленточного конвейера с целью стабилизации погонной нагрузки разработаны модели конвейера и привода с асинхронным электродвигателем с фазным ротором. Эти объекты представлены простейшими типовыми моделями первого и второго порядка. Для них рассчитаны коэффициенты передачи, постоянные времени и постоянные запаздывания. В работе рассмотрены различные варианты регулирования при помощи типовых П-, ПД- и ПИД-регуляторов, настройки которых получены с~использованием программы оптимизации, входящей в состав вычислительных средств “МАСС”. Разработанный алгоритм управления скоростью движения конвейера позволяет определять значение необходимой скорости движения, осуществлять формирование управляющего сигнала в~зависимости от прогнозируемых характеристик случайного грузопотока, от отклонения реальной скорости от заданной и от текущей погонной нагрузки.

Существенным недостатком этой работы является то, что модели элементов автоматической системы излишне упрощены, синтез проведен также самыми упрощенными способами, а~настройки регуляторов получены для конкретного грузопотока, ленточного конвейера и привода, что не позволяет применять эту систему управления для грузопотоков других шахт и карьеров.\\

В работе \textbf{Сухарева И. А.} \textit{«Управление конвейерными линиями на базе асинхронного электропривода в рамках АСУ ТП»} \cite{isukharev} показано, что эффективность работы  конвейерного транспорта существенно повышается с применением асинхронного электропривода, который позволяет обеспечить высокий уровень качества работы и надежность всей системы. Сделанный анализ исполнительных элементов показал целесообразность использования асинхронного двигателя с~фазным ротором, включенного по схеме асинхронно-вентильного каскада. Автором получена математическая модель асинхронного двигателя в виде нелинейной системы дифференциальных уравнений первого порядка, описывающая разомкнутую систему электропривода, выбран способ регулирования скорости асинхронного двигателя, базирующегося на импульсном управлении. В~работе предложен модифицированный алгоритм синтеза системы управления многодвигательным приводом, которая должна обеспечить компенсацию механических колебаний, вызванных упругостью ленты и синхронизацию работы исполнительных элементов. Система синхронизации реализуется по принципу дистанционного электрического вала с одним ведущим исполнительным элементом и $n - 1$  ведомыми. Для оптимальной стабилизации возмущающих воздействий на привод со стороны ленты конвейера выбран интегральный критерий
$$
\int\limits_0^\infty [M(t) - M_{BH} (t)]^2 dt,
$$
где $M(t)$ -- момент, развиваемый приводом в каждом канале управления, $M_{BH} (t)$ -- внешний момент сопротивления конвейера движению. Скачкообразное изменение внешнего момента соответствует переводу конвейера на другую скорость.
Недостатком работы является то, что автором не рассматривается задача загрузки конвейера, который в данной работе рассматривается как возмущающее воздействие.\\

Целью работы \textbf{Сокотнюка Ю.А.} \textit{«Система автоматического управления наклонным ленточным конвейером»} \cite{usokotnuk} являлась разработка системы автоматического управления наклонным ленточным конвейером с регулируемым электроприводом, обеспечивающим ограничение динамических нагрузок привода. Это достигается  введением обратной связи по разности скоростей ленты на приводном и хвостовом барабанах, использовании общего регулятора скорости вращения и тяговой способности второго барабана, зависящей от режима работы первого барабана, что приводит к выравниванию тяговой способности обоих барабанов. 

Автором рассмотрены технологические схемы ленточных конвейеров с одним и двумя приводами и их различным расположением (на головном и хвостовом барабане, на порожней ветви, второй привод у головного барабана), получены структурные схемы этих конвейеров и получены их модели в виде передаточных функций второго порядка:
$$
W(p) = \frac{V(p)}{M_d(p)} = \frac{T_1^2p + T_2p + 1}{T_mp(T_3^2p + T_4p + 1)}.
$$
Автор предложил повысить демпфирующие способности электропривода путем введения на вход регулятора скорости дополнительной отрицательной обратной связи по разности скоростей движения ленты на приводном и хвостовом барабанах. Разработан вариант системы автоматического управления с коррекцией по величине проскальзывания ленты на приводном барабане ленточного конвейера, в которой величина максимального движущего момента, развиваемого приводом, уменьшается в зависимости от величины проскальзывания, обеспечивая таким образом тяговую способность привода.

Недостатком такой схемы управления является то, что для стабилизации тяговой способности привода автор предлагает изменять скорость вращения привода, тем самым усложняя закон регулирования, включая в него излишние пуско-тормозные режимы, а не использовать при решении данной задачи натяжное устройство конвейера. 

Автором предложена схема системы автоматического управления с тиристорным электроприводом, которая может обеспечить управление скоростью ленточного конвейера в~зависимости от~уровня загрузки. Использование такой схемы регулирования возможно только при расположении привода в хвостовой части конвейера, так как в любом другом случае конвейер, являясь инерционным звеном, существенно задержит отработку сигнала задания, что приведет к ссыпанию груза с ленты. Кроме того, для этой системы не разработаны ни~алгоритмы управления, ни~структура регулирующего устройства, что делает затруднительным ее использование.\\

В работе \textbf{Гершуна С. В.} \textit{«Система автоматической стабилизации тягового фактора магистрального ленточного конвейера с двухдвигательным приводом»} \cite{sgershun} разработана модель системы стабилизации тягового фактора магистрального ленточного конвейера с двухдвигательным приводом. Это обеспечивает  беспробуксовочную работу привода конвейера, что, в свою очередь, уменьшает динамические усилия в ленте и снижает коэффициент запаса по прочности конвейерной ленты, позволяя выбирать для конвейера менее дорогие типы ленты. В диссертации разработаны модели конвейера с двухдвигательным приводом и натяжным устройством, расположенным в хвостовой части, и короткозамкнутых частотно-управляемых приводов. Модели представлены в виде системы дифференциальных уравнений и реализованы в приложении Simulunk пакета MatLab. 

Также в работе были реализованы:
\begin{itemize}
\item модель частотно-векторной системы управления электроприводом, изначально предложенной фирмой Siemens. Система построена в виде двух каналов: канала стабилизации потокосцепления ротора и канала управления скоростью вращения ротора, каждый из которых в свою очередь является системой подчиненного регулирования;
\item регулятор натяжения. Была найдена зависимость величины тягового фактора веса натяжного устройства. Также была найдена зависимость перемещения натяжного устройства от веса его грузов. В регуляторе натяжения сравнивается вес натяжных грузов, соответствующий текущей величине тягового фактора и заданной. Их разница подается на вход ПИ-регулятора и преобразуется в соответствующее значение добавочного перемещения каретки натяжного устройства. 
\end{itemize}

В итоге была получена математическая модель, которую можно использовать для дальнейшего изучения двухдвигательного конвейера, его моделирования при различных условиях, а~также разработки систем управления. Рассчитаны мощности двигателей электропривода для~конвейера с заданными параметрами. Двигатели промоделированы на основе характеристик существующих двигателей с аналогичными мощностями. Получена модель автоматической стабилизации величины тягового фактора магистрального конвейера с двухдвигательным приводом. Автоматическая стабилизация осуществляется регулированием положения каретки натяжного устройства, что обеспечивает приемлемое соотношение натяжений в~грузовой и~порожней ветвях ленты конвейера.\\

В работе \textbf{Серикова С. А.} \textit{«Оптимальная адаптивная система управления электроприводами подвесных конвейеров»} \cite{sserikov} разработана оптимальная адаптивная система автоматического управления электроприводами подвесных конвейеров с целью минимизации колебаний грузов относительно положения равновесия независимо от характера их нестационарности и воздействия внешних возмущающих факторов.

В работе решены следующие задачи:
\begin{itemize}
\item разработана математическая модель электромеханической системы конвейера с приводом переменного и постоянного тока, учитывающая распределено-упругий характер параметров механической части;
\item разработан алгоритм адаптивной идентификации динамических объектов;
\item проведены исследования математической модели электромеханической системы приводов конвейера, подтверждающие высокою скорость сходимости и точность полученных оценок разработанного алгоритма адаптивной идентификации динамических объектов;
\item разработано программное обеспечение, позволяющее идентифицировать параметры динамических объектов для получения их математических моделей;
\item разработана структура аппаратной части и алгоритмы функционирования адаптивной системы автоматического управления приводами подвесных конвейеров;
\item проведены исследования полученных математических моделей.
\end{itemize}

В работе \textbf{Черемушконой М. С.} \textit{«Синтез алгоритмов управления многодвигательным электроприводом конвейерного транспорта с использованием полупроводниковых преобразователей»} \cite{mcheremushkina} создан алгоритм управления с коррекцией сигналов задания в системе управления многодвигательным частотно-регулируемым приводом конвейера, учитывающим случайный характер грузопотока. Установлены зависимости изменения электромагнитного момента  двигателя от динамических нагрузок на валу привода конвейера в режиме пуска и в рабочих режимах при различных алгоритмах управления, позволяющие обеспечить необходимый алгоритм функционирования системы управления многодвигательным частотно-регулируемым приводом конвейера, что обеспечивает энергетически эффективный режим работы транспортной установки. В работе рассмотрены возможности построения микропроцессорной системы управления комплексом конвейерных линий. Приводятся технические требования к аппаратным средствам и организации связи системы верхнего уровня и локальных систем управления. Также рассмотрена организация и структура системы контроля и диагностики комплекса приводов конвейерной линии. 

Целью работы являлось повышение энерго- и ресурсосбережения конвейерного транспорта путем реализации разработки алгоритмов управления частотно-регулируемыми приводами конвейера. 

В работе решены следующие задачи:
\begin{itemize}
\item разработана математическая модель системы Привод – Конвейер  и поточно-транспортной системы из нескольких параллельно и последовательно включенных конвейеров;
\item разработаны алгоритмы управления частотно-регулируемым приводом конвейерного транспорта, позволяющие обеспечить равномерность натяжения ленты по ее длине и автоматическое регулирование отдельных двигателей с целью равномерного распределения нагрузки между ними;
\item проведено определение эффективности применения разработанных алгоритмов управления.
\end{itemize}

В диссертации  на основе выполненных теоретических и экспериментальных исследований решена актуальная практическая задача повышения энерго- и ресурсосбережения конвейерного транспорта путем реализации разработанных алгоритмов и системы управления асинхронным частотно-регулируемым многодвигательным электроприводом конвейерного транспорта. Разработанная математическая модель электромеханической системы конвейер – многодвигательный асинхронный частотно-регулируемый электропривод с реализацией в среде Simulink пакета программ MatLab позволяет выполнить исследования режимов работы электропривода конвейера с~учетом специфики работы механизма при различных алгоритмах управления электроприводом.

Также в работе показана целесообразность использования локального цифрового управления моментом асинхронного двигателя с разработанным алгоритмом корректировки сигналов задания, обеспечивающего повышение равномерности распределения нагрузки между приводными двигателями и ограничение динамических нагрузок на ленту.\\

В работе \textbf{Мартынова В. В.} \textit{«Разработка алгоритмов и устройств автоматического контроля количества сыпучих материалов на ленте конвейера на основе деформации электрополей»} \cite{vmartynov} разработан метод автоматического оперативного контроля количества сыпучего материала на ленте конвейера, помещенного в однородное электрическое поле, математическая модель и алгоритмы определения параметров искажения электрического поля, созданы технические средства контроля на этой основе.

Лента конвейера с транспортируемым материалом пропускается сквозь полость измерительного конденсатора, одна из пластин которого секционирована с целью создания в исследуемом пространстве квазистационарного однородного электрического поля. Изменения характеристик этого поля, порождаемые транспортируемым материалом, с высокой точностью измеряются, что~позволяет вычислить необходимые параметры грузопотока. 

Основные научные положения работы состоят в следующем:
\begin{itemize}
\item разработанный способ формирования квазистационарного однородного электрического поля позволяет оценивать параметры, приводящие к его возмущению;
\item найдена математическая модель оценки параметров электрического поля, позволяющая создать методику оперативного измерения параметров грузопотока на ленте конвейера;
\item предложенная концепция построения преобразователя позволяет формировать схемы различных измерителей параметров грузопотока на ленте конвейера.
\end{itemize}

Работа \textbf{Городецкого А. В.} \textit{«Исследование влияния пускового режима на выбор параметров линейной части ленточного конвейера»} \cite{agorodetsky} посвящена исследованию режима пуска ленточного конвейера на ходовых опорах с целью разработки методики расчета параметров его линейной части, что обеспечивает его надежную работу во всех режимах транспортировки грузов. 

В работе исследованы методы, которые путем установления рациональных параметров линейной части конвейера обеспечивают режим пуска, при котором натяжения в ленте не превышают допустимых. Автором разработаны следующие научные положения:
\begin{itemize}
\item установлены зависимости для определения предварительного натяжения цепного контура горизонтального ленточного конвейера, отличающиеся тем, что они учитывают падение натяжения цепей у головной звездочки при пуске конвейера за счет сил трения упругого скольжения ленты по траверсам ходовых опор; 
\item разработаны методики тягового расчета и расчета основных параметров линейной части ленточного конвейера, позволяющие на стадии проектирования выбирать ширину ленты, определять предварительное натяжение цепей с учетом падения натяжения у головной звездочки при пуске конвейера, длину цепного контура по допустимому уровню натяжений в цепях с учетом сил, возникающих при пуске конвейера.
\end{itemize}

В работе \textbf{Авдиенко И. Н.} \textit{«Автоматизированное управление многоярусной конвейерной системой с композиционными полимерными лентами»} \cite{iavdienko} представлен аналитический обзор конвейерных систем, в котором описаны особенности конвейеров, различающихся по областям применения, по назначению и другим факторам. Описаны основные направления развития конвейерных систем.  Особое внимание в обзоре уделено ленточным конвейерам, техническим особенностям их узлов, в частности, особенностям ленты. 

В работе проведено моделирование процесса деформации композитной полимерной ленты, в~ходе которого рассмотрены физико-механические особенности ленты такого типа, определен вид основных зависимостей, определяющих деформационные свойства ленты. Также разработана программная реализация математической модели деформации ленты, позволяющая получить описание статических и динамических свойств ленты. 

В продолжение исследований было проведено моделирование конвейерной системы с композитными полимерными лентами. В ходе моделирования проведен анализ приводов, используемых в конвейерных системах,  разработана математическая модель системы, состоящая из двух модулей формирования грузового и тягово-несущего потоков. Модель учитывает весь комплекс особенностей конвейерной системы, включая зависимость силы сопротивлению движению от~деформирования транспортируемого груза. 

На основе модели конвейерной системы разработана методика тягового расчета, позволяющая выбрать величину начального усилия натяжения ленты с учетом ее деформируемости. 

В заключение рассмотрены способы автоматизированного управления конвейерными системами, при этом были выбраны критерии управления многоярусными конвейерными системами, обоснована необходимость адаптации математической модели конвейерной системы в условиях дрейфа параметра ленты, обусловленного процессами старения. Разработан алгоритм адаптации математической модели. Сформулирована задача системы управления производительностью конвейерной системы. Разработана схема и алгоритм адаптивного управления конвейерной системой на основе разработанной математической модели и принципа управления по возмущению.\\

В работе \textbf{ Кангина В. В.} \textit{«Повышение эффективности систем управления распределительными конвейерами и автоматизированными складами на основе структурного моделирования процессов и объектов»} \cite{vkangin} рассмотрены различные типы конвейерных установок и систем, показаны примеры, методы управления, рассмотрены основные проблемы при организации работы конвейеров. Рассмотрены три способа реализации систем управления конвейерными системами -- аппаратный, программный и программно-аппаратный. Указаны достоинства и недостатки каждого из способов. В частности сказано, что аппаратный способ построения конвейерных систем нашел применение только при автоматизации простейших конвейерных установок с жестким циклом, постоянными маршрутами изделий и небольшим числом адресатов.

Большое внимание уделено программным методам реализации автоматизированных систем, построению с их помощью как простых, так и иерархических систем автоматизации.

В работе дана научно обоснованная методология построения иерархических систем управления, используемых в системах автоматизации транспортных и складских комплексов, а также конвейерных линий. Эта методология включает в себя такие аспекты как декомпозиция технологического процесса на подпроцессы, построение и исследование терминальных моделей объектов, формулирование задач, решаемых на различных уровнях систем управления, синтез структуры элементов нижнего уровня, определение числа уровней в системе. Построены терминальные модели таких объектов автоматизации как распределительный конвейер, конвейерная система модульного типа, исследование которых позволило сформулировать алгоритмы функционирования элементов нижнего уровня систем управления. 

Предложен новый подход к построению систем управления иерархического типа, позволяющий отобразить свойства объекта управления в оптимальную структуру системы управления, характеризующуюся максимальным быстродействием, простыми алгоритмами управления, минимальными аппаратными затратами. 

Результаты проведенных в работе исследований могут быть использованы в качестве методологической основы для научно-обоснованного проектирования эффективных систем управления не только распределительными конвейерами и складскими комплексами, но и другими типами систем управления. Результаты могут быть использованы для разработки систем управления промышленными объектами различного технологического назначения.\\

В работе \textbf{ Симонса А. } \textit{«Разработка и научное обоснование параметров тормозного устройства мощных наклонных ленточных конвейеров»} \cite{asimons} рассмотрены различные виды тормозных устройств, применяемых в транспортных системах горной промышленности. В работе установлены закономерности формирования тормозного момента для тормоза с осевым нажатием, размещаемого на валу приводного барабана ленточного конвейера, разработаны методы расчета и выбора параметров тормозного устройства. В частности, разработана принципиальная схема тормоза с осевым нажатием и увеличенным тормозным моментом, установлены закономерности формирования тормозного усилия  в тормозах с осевым нажатием и разработана математическая модель тормозного устройства. В соответствии с разработанной моделью требуемое тормозное усилие на ободе приводного барабана может быть найдено из дифференциального уравнения:

$$ \Big( m_k + \frac{2cJ_1 i^2_p \eta}{D^2} \Big) \frac{dV}{dt} + \Sigma W + W_T = 0, $$

где $m_k$ -- приведенная масса поступательно движущихся элементов конвейера, кг; $c$ -- коэффициент учета моментов инерций остальных вращающихся элементов привода, кроме вала привода конвейера; $J_1$ -- момент инерции вращающихся масс вала привода конвейера; $i_p$ -- передаточное отношение редуктора привода конвейера; $\eta$ -- КПД передаточного механизма привода; $D$ -- диаметр приводного барабана, м; $\frac{dV}{dt}$ -- производная скорости при замедлении ленты в процессе торможения; $ \Sigma W $ -- суммарные статические сопротивления движению ленты; $W_T$ -- требуемое тормозное усилие.Из этого уравнения следует, что требуемый тормозной момент на валу приводного барабана для торможения конвейера равен:

$$ M_T = \Big( ( m_k + \frac{2cJ_1 i^2_p \eta}{D^2} )\frac{V}{t} - \Sigma W \Big) \frac{D}{2}, $$

где $V$ -- начальная скорость движения ленты, м/с; $t$ -- время торможения, с.

\section{Постановка задач} \label{sect1_2}

Анализ рассмотренных в обзоре работ показывает, что проблема снижения износа ленты при~останове ленточного конвейера не имеет окончательного решения. В настоящее время в связи с развитием промышленной микропроцессорной управляющей техники появилась возможность решения этой проблемы современными методами. Это позволяет сформулировать следующие задачи исследования в области автоматизации и управления конвейерными установками:
\begin{itemize}
\item  Исследование существующей математической модели ленточного конвейера, описанной в~\cite{vdmitrieva1}, и ее доработка для последующих исследований эксплуатации конвейерной установки в режимах пуска и торможения;
\item Исследование переходных процессов пуска и торможения конвейера и поиск способов управления этими процессами с целью оптимизации и повышения эффективности работы конвейера;
\item Разработка и исследование алгоритмов управления, оптимизирующих переходные процессы пуска и торможения конвейера. Эти алгоритмы управления должны обеспечивать номинальные значения натяжений в грузовой и порожней ветвях конвейера для исключения проскальзывания ленты на приводном барабане при торможении или свободном выбеге конвейера, а после его останова исключать провисание ленты между роликоопорами. Это~позволит уменьшить износ ленты, что, в свою очередь, позволит сократить расходы на~обслуживание и ремонт, продлит срок службы конвейерной установки;
\item Разработка структуры и варианта реализации комплексной системы автоматического управления ленточным конвейером, в которой реализуются алгоритмы управления процессами пуска и торможения конвейера, а также алгоритмы управления, описанные в работе~\cite{vdmitrieva}. Эта задача также включает в себя разработку программного обеспечения и подбор аппаратного обеспечения для построения комплексной системы автоматического управления.
\end{itemize}
