\chapter*{Заключение}						% Заголовок
\addcontentsline{toc}{chapter}{Заключение}	% Добавляем его в оглавление

В результате проведенных исследований дано новое решение актуальной научно-технической задачи разработки алгоритма предварительного управляемого торможения магистрального ленточного конвейера, а также решение задачи реализации алгоритмов автоматического управления магистральным ленточным конвейером (как алгоритма, разработанного в настоящей работе, так и алгоритмов, разработанных ранее в работе \cite{vdmitrieva}) в виде разработки структуры и варианта реализации комплексной автоматизированной системы управления магистральным ленточным конвейером. Представленные решения позволяют повысить эффективность эксплуатации ленточных конвейеров в различных режимах работы с исключением или минимизацией проскальзывания ленты конвейера на приводном барабане и существенным снижением динамических усилий в ленте в переходных режимах. 

Основные научные и практические выводы, полученные лично автором:
\begin{enumerate}
  \item Эффективность эксплуатации конвейерных устройств на горных предприятиях может быть повышена только при условии автоматического управления технологическими операциями, выполняемыми в процессе эксплуатации. Для создания автоматической системы управления конвейерной установкой, позволяющей исключить проскальзывание ленты конвейера на приводном барабане и снизить динамические усилия в ленте в переходных режимах, необходимы управляющие алгоритмы, учитывающие механические свойства  конвейерной ленты и особенности ее поведения: распространение упругих волн, характер возникновения напряжений и деформаций и сил сопротивления движению на различных участках ленты;
  \item Произведена доработка существующей математической модели ленточного конвейера. Доработанная математическая модель позволяет исследовать переходные процессы, возникающие при торможении приводного и хвостового барабанов конвейера;
  \item Модельные исследования показали, что величина тягового фактора ленточного конвейера при его останове в большинстве случаев превышает технологическое значение, определяемое параметрами конвейера (не соблюдается условие Эйлера), что приводит к проскальзыванию ленты на приводном барабане. На основе модельных исследований был разработан алгоритм управляемого торможения ленточного конвейера, использующий тормозные устройства, расположенные на приводном и на хвостовом барабанах;
  \item Разработана методика расчета параметров алгоритма управляемого торможения ленточного конвейера, которая позволяет определить требуемые величины тормозных моментов и тягового фактора, исходя из технологических параметров используемого конвейера;
  \item Исследование полученного алгоритма совместно с доработанной математической моделью ленточного конвейера показало, что применение этого алгоритма позволяет минимизировать, а в большинстве случаев исключить проскальзывание ленты при торможении конвейера, а также более чем в шесть раз сократить временной интервал, в течении которого проскальзывание ленты на приводном барабане гипотетически может возникнуть;
  \item Наилучшие результаты повышения эффективности эксплуатации конвейерного транспорта могут быть достигнуты только при совместном использовании алгоритмов, осуществляющих не только стабилизацию тягового фактора при торможении конвейера, но и управление другими процессами -- стабилизацией тягового фактора в номинальном режиме работы конвейера и регулированием скорости движения ленты при переключении для снижения динамических нагрузок. Для этих целей была разработана структура комплексной автоматизированной системы управления ленточным конвейером, включающая в себя указанные алгоритмы. Предложен вариант реализации системы управления, включающий в себя выбор и обоснование аппаратного обеспечения и компонентов системы, а также разработку программного обеспечения, реализующего указанные управляющие алгоритмы на современном промышленном языке программирования, соответствующем стандарту МЭК 61131-3.
\end{enumerate}

\clearpage