\chapter{Вёрстка таблиц} \label{chapt3}

\section{Таблица обыкновенная} \label{sect3_1}

Так размещается таблица:

\begin{table} [htbp]
  \centering
  \parbox{15cm}{\caption{Название таблицы}\label{Ts0Sib}}
%  \begin{center}
  \begin{tabular}{| p{3cm} || p{3cm} | p{3cm} | p{4cm}l |}
  \hline
  \hline
  Месяц   & \centering $T_{min}$, К & \centering $T_{max}$, К &\centering  $(T_{max} - T_{min})$, К & \\
  \hline
  Декабрь &\centering  253.575   &\centering  257.778    &\centering      4.203  &   \\
  Январь  &\centering  262.431   &\centering  263.214    &\centering      0.783  &   \\
  Февраль &\centering  261.184   &\centering  260.381    &\centering     $-$0.803  &   \\
  \hline
  \hline
  \end{tabular}
%  \end{center}
\end{table}

%\newpage
%============================================================================================================================

\section{Параграф - два} \label{sect3_2}

Некоторый текст.

%\newpage
%============================================================================================================================

\section{Параграф с подпараграфами} \label{sect3_3}

\subsection{Подпараграф - один} \label{subsect3_3_1}

Некоторый текст.

\subsection{Подпараграф - два} \label{subsect3_3_2}

Некоторый текст.

\clearpage

%\newpage
%============================================================================================================================

\section{Ссылки} \label{sect1_2}
Сошлёмся на библиографию: \cite{bib1}, \cite{bib2}, \cite{bib3,bib4,bib5}.

Сошлёмся на приложения: Приложение \ref{AppendixA}, Приложение \ref{AppendixB2}.

Сошлёмся на формулу: формула (\ref{eq:equation1}).

Сошлёмся на изображение: рисунок \ref{img:knuth}.

%\newpage
%============================================================================================================================

\section{Формулы} \label{sect1_3}

\subsection{Ненумерованные одиночные формулы} \label{subsect1_3_1}

Вот так может выглядеть формула, которую необходимо вставить в строку по тексту: $x \approx \sin x$ при $x \to 0$.

А вот так выглядит ненумерованая отдельностоящая формула c подстрочными и надстрочными индексами:
$$
(x_1+x_2)^2 = x_1^2 + 2 x_1 x_2 + x_2^2
$$

При использовании дробей формулы могут получаться очень высокие:
$$
  \frac{1}{\sqrt(2)+
  \displaystyle\frac{1}{\sqrt{2}+
  \displaystyle\frac{1}{\sqrt{2}+\cdots}}}
$$

В формулах можно использовать греческие буквы:
$$
\alpha\beta\gamma\delta\epsilon\varepsilon\zeta\eta\theta\vartheta\iota\kappa\lambda\\mu\nu\xi\pi\varpi\rho\varrho\sigma\varsigma\tau\upsilon\phi\varphi\chi\psi\omega\Gamma\Delta\Theta\Lambda\Xi\Pi\Sigma\Upsilon\Phi\Psi\Omega
$$

%\newpage
%============================================================================================================================

\subsection{Ненумерованные многострочные формулы} \label{subsect1_3_2}

Вот так можно написать две формулы, не нумеруя их, чтобы знаки равно были строго друг под другом:
\begin{eqnarray}
  f_W & = & \min \left( 1, \max \left( 0, \frac{W_{soil} / W_{max}}{W_{crit}} \right)  \right), \nonumber \\
  f_T & = & \min \left( 1, \max \left( 0, \frac{T_s / T_{melt}}{T_{crit}} \right)  \right), \nonumber
\end{eqnarray}

Можно использовать разные математические алфавиты:
\begin{eqnarray}
\mathcal{ABCDEFGHIJKLMNOPQRSTUVWXYZ} \nonumber \\
\mathfrak{ABCDEFGHIJKLMNOPQRSTUVWXYZ} \nonumber \\
\mathbb{ABCDEFGHIJKLMNOPQRSTUVWXYZ} \nonumber
\end{eqnarray}

Посмотрим на систему уравнений на примере аттрактора Лоренца:

$$
\left\{
  \begin{array}{rl}
    \dot x = & \sigma (y-x) \\
    \dot y = & x (r - z) - y \\
    \dot z = & xy - bz
  \end{array}
\right.
$$

А для вёрстки матриц удобно использовать многоточия:
$$
\left(
  \begin{array}{ccc}
  	a_{11} & \ldots & a_{1n} \\
  	\vdots & \ddots & \vdots \\
  	a_{n1} & \ldots & a_{nn} \\
  \end{array}
\right)
$$


%\newpage
%============================================================================================================================
\subsection{Нумерованные формулы} \label{subsect1_3_3}

А вот так пишется нумерованая формула:
\begin{equation}
  \label{eq:equation1}
  e = \lim_{n \to \infty} \left( 1+\frac{1}{n} \right) ^n
\end{equation}

Нумерованых формул может быть несколько:
\begin{equation}
  \label{eq:equation2}
  \lim_{n \to \infty} \sum_{k=1}^n \frac{1}{k^2} = \frac{\pi^2}{6}
\end{equation}

В последствии на формулы (\ref{eq:equation1}) и (\ref{eq:equation2}) можно ссылаться.

%\newpage
%============================================================================================================================

\clearpage

\chapter{Длинное название главы, в которой мы смотрим на примеры того, как будут верстаться изображения и списки} \label{chapt2}

\section{Одиночное изображение} \label{sect2_1}

\begin{figure} [h] 
  \center
  \includegraphics [scale=0.27] {LaTeX}
  \caption{TeX.} 
  \label{img:latex}  
\end{figure}

%\newpage
%============================================================================================================================
\section{Длинное название параграфа, в котором мы узнаём как сделать две картинки с общим номером и названием} \label{sect2_2}

А это две картинки под общим номером и названием:
\begin{figure}[h]
  \begin{minipage}[h]{0.49\linewidth}
    \center{\includegraphics[width=0.5\linewidth]{knuth1} \\ а)}
  \end{minipage}
  \hfill
  \begin{minipage}[h]{0.49\linewidth}
    \center{\includegraphics[width=0.5\linewidth]{knuth2} \\ б)}
  \end{minipage}
  \caption{Очень длинная подпись к изображению, на котором представлены две фотографии Дональда Кнута}
  \label{img:knuth}  
\end{figure}

%\newpage
%============================================================================================================================
\section{Пример вёрстки списоков} \label{sect2_3}

\noindent Нумерованный список:
\begin{enumerate}
  \item Первый пункт.
  \item Второй пункт.
  \item Третий пункт.
\end{enumerate}

\noindent Маркированный список:
\begin{itemize}
  \item Первый пункт.
  \item Второй пункт.
  \item Третий пункт.
\end{itemize}

\noindent Вложенные списки:
\begin{itemize}
  \item Имеется маркированный список.
  \begin{enumerate}
    \item В нём лежит нумерованный список,
    \item в котором
    \begin{itemize}
      \item лежит ещё один маркированный список.
    \end{itemize}    
  \end{enumerate}
\end{itemize}


\clearpage